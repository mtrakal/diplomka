\begin{comment}
2013-01-21 http://cs.wikipedia.org/wiki/Cloud_computing
2012-12-11 http://www.lupa.cz/clanky/co-je-a-co-neni-cloud/

- historie
2013-01-29 http://technik.ihned.cz/c1-48480960-miri-pocitace-do-oblak
2013-02-20 http://www.itbiz.cz/cloud-computing-v-praxi-maly-pohled-do-historie-aneb-vse-co-jste-o-nem-chteli-vedet-ale-bali-jste-se-zeptat
2013-02-23 http://www.ddconnect.cz/brezen-2012/datova-centra.html


účtování po menších úsecích než u hostingu, např. hodinové
dynamické zvýšení výkonu když je potřeba - virtualizace - denní zátěž, noční klid, úspora financí? Možná

Cloudy:
AWS - Amazon Web Services
App Engine - Google
Azure - MS
OnLive - cloudová vzdálená plocha NX

http://www.zdrojak.cz/clanky/zaciname-s-node-js-na-windows-azure/
https://developers.google.com/appengine/
http://mtdiplomka.appspot.com/

2013-03-12 http://www.systemonline.cz/virtualizace/
2013-03-12 http://en.wikipedia.org/wiki/Cloud_computing_security
2013-03-12 http://www.infoworld.com/d/security-central/gartner-seven-cloud-computing-security-risks-853
2013-03-12 http://www.technologyreview.com/featuredstory/416804/security-in-the-ether/
http://www.cloud.cz/bezpenost/175-role-bezpecnosti-v-duveryhodnem-cloudu.html
http://www.itbiz.cz/cloud-computing-v-praxi-cloud-a-bezpecnost-2
2013-05-07 http://www.lupa.cz/clanky/predstaveni-cloudovych-sluzeb-amazon-web-services/
2013-05-07 http://www.zdrojak.cz/clanky/pouzivame-datove-uloziste-amazon-s3/

kompilace do HTML: htlatex Diplomka.tex html "" -dhtml "--interaction=nonstopmode"


Poznámky k praktický:
dostupnost dat z venku (mimo net 24/7)
DB, přístup k filesystému (zpřístupnění intranetu - priv. cloud), intranetové aplikace (obchoďák...)
hW nároky instalce služeb, cena, HW, čas k nasazení, hloubka znalostí pro implementátora (školení), tabulka spec./bc/ing/...školení
čeština

http://www.microsoft.com/cs-cz/server-cloud/windows-azure/default.aspx

Link po dokončení instalace: http://www.windowsazure.com/en-us/documentation/articles/web-sites-dotnet-get-started/

\end{comment}
